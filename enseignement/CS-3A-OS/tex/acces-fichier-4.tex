\documentclass[margin=1mm]{standalone}
% %%%%%%%%%%%%%%%%%%%%%%%%%%%%%%%%%%%%%%%%%%%%%%%%%%%
% %%%%%%%%%%%%%%%%%%%%%%%%%%%%%%%%%%%%%%%%%%%%%%%%%%%
% le package de la langue du document
\usepackage[french]{babel}
% le package du codage du document
\usepackage[utf8]{inputenc}
% le package du style de la font
\usepackage[default,semibold]{/Users/idiraitsadoune/Desktop/Teaching-Desktop/slides-cours-td-tp-conference/0-beamer-templates-v3/sourcesanspro/tex/sourcesanspro}
\usepackage{fontspec}
\setmainfont{SourceSansPro}
% le package pour la font de \ttfamily
\usepackage{fontspec}
\setmonofont{Monaco}
% le package pour la font Math
\usepackage{amsmath}
\usepackage[math-style=TeX]{unicode-math}

% pour le dessin
\usepackage{fontawesome}
\usepackage{tikz}
\usetikzlibrary{arrows.meta,automata,positioning}
\usetikzlibrary{arrows.meta}
\usetikzlibrary{decorations.pathreplacing}
\usetikzlibrary{shapes}

% %%%%%%%%%%%%%%%%%%%%%%%%%%%%%%%%%%%%%%%%%%%%%%%%%%%
% %%%%%%%%%%%%%%%%%%%%%%%%%%%%%%%%%%%%%%%%%%%%%%%%%%%
% The main document 
\begin{document}     
% %%%%%%%%%%%%%%%%%%%%%%%%%%%%%%%%%%%%%%%%%%%%%%%%%%%
% %%%%%%%%%%%%%%%%%%%%%%%%%%%%%%%%%%%%%%%%%%%%%%%%%%%
% %%%%%%%%%%%%%%%%%%%%%%%%%%%%%%%%%%%%%%%%%%%%%%%%%%%
% %%%%%%%%%%%%%%%%%%%%%%%%%%%%%%%%%%%%%%%%%%%%%%%%%%%


\begin{tikzpicture}[ultra thick]

    \node at (0,10) { }; % pour pas que ça saute
		
    % Application
    \draw[color=blue,fill=blue!20] (-12,8.5) rectangle (-4,6.5);
    \node at (-8,9) {\textbf{Application}};
    
    % RAM
    \draw[color=blue,fill=blue!20] (-12,4) rectangle (-4,0);
    \node at (-8,4.5) {\textbf{RAM (OS)}};
    \draw[color=red,fill=red!20] (-11,3.5) rectangle (-9,1);
    \node at (-10,.5) {Cache répertoires};
    \draw[color=red,fill=red!20] (-7,3.5) rectangle (-5,1);
    \node at (-6,.5) {Table fichiers ouverts};
    
    % le disque
    \draw[color=blue,fill=blue!20] (0,0) rectangle (6,5);
    \draw[color=blue,fill=blue!20] (3,0) ellipse (3 and .8);
    \draw[color=blue,fill=blue!20] (3,5) ellipse (3 and .8);
    \node at (3,6.3) {\textbf{Disque}};
    \node[draw,color=red,fill=red!40,text=black,minimum width=2cm,minimum height=1cm] (dir) at (2,3.5) {\small Répertoire};
    \node[draw,color=red,fill=red!40,text=black,minimum width=1cm,minimum height=1cm] (fcb) at (4.5,3.5) {\small FCB};
    \draw[color=red,fill=red!20] (.8,2.7) rectangle (5.2,1.3);
    \node[draw,color=red,fill=red!40,text=black,minimum width=1cm,minimum height=1cm] (file1) at (1.5,2) {\small Bloc};
    \node[draw,color=red,fill=red!40,text=black,minimum width=1cm,minimum height=1cm] (file2) at (3,2) {\small Bloc};
    \node[draw,color=red,fill=red!40,text=black,minimum width=1cm,minimum height=1cm] (file3) at (4.5,2) {\small Bloc};
    \node at (4,1.1) {\textit{Fichier}};

    \draw[fill=red] (2.7,3) rectangle (2.9,4);
    \draw[-Stealth, ultra thick] (2.8,4) to (3.2,4.6) to (4.2,4.6) to (fcb);
    \draw[-Stealth, ultra thick] (fcb) to (file1);
    \draw[-Stealth, ultra thick] (fcb) to (file2);
    \draw[-Stealth, ultra thick] (fcb) to (file3);

    \node[draw,fill=red!40,minimum width=1.5cm,minimum height=.8cm] (fcb-dir) at (-6,2.8) {\small FCB(dir)};
    \draw[-Stealth, ultra thick] (fcb-dir) to (dir);

    \node at (-9,8) {\texttt{\small open(dir/file.ext)}};

    %\onslide<4-11>{
        \node at (2,9.5) {\begin{minipage}{6cm}
            \textbf{Lors du premier accès\\ à~un~fichier\ldots}
            \end{minipage}};
    %
    
\end{tikzpicture}



% %%%%%%%%%%%%%%%%%%%%%%%%%%%%%%%%%%%%%%%%%%%%%%%%%%%
% %%%%%%%%%%%%%%%%%%%%%%%%%%%%%%%%%%%%%%%%%%%%%%%%%%% 
\end{document}
% %%%%%%%%%%%%%%%%%%%%%%%%%%%%%%%%%%%%%%%%%%%%%%%%%%%
% %%%%%%%%%%%%%%%%%%%%%%%%%%%%%%%%%%%%%%%%%%%%%%%%%%%